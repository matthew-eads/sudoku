\documentclass{article}
\pagestyle{plain}
\usepackage{sectsty}
\usepackage{titling}
\usepackage[utf8]{inputenc}
\usepackage{multirow}
\usepackage{array}
\usepackage[english]{babel}

\setlength{\droptitle}{-5em}
\sectionfont{\fontsize{12}{15}}
\begin{document}
\title{Backtracking Sudoku Solver}
\date{March 12, 2015}
\author{Matthew Eads}
\maketitle
\section{Introduction}
Many features of the sudoku solver used in this assignment were given
by Prof. Anselm Blumer; the framework for reading in and printing out the sudoku
puzzle, as well as the framework for solving were written for us.
Thus the focus of this assignment was writing and experimenting with
different heuristics to optimize the backtracking algorithm. Algorithms were
outlined by Prof. Blumer, and Russel \&{} Norvig's \textit{Artificial Intelligence:
A Modern Approach}.

The different heuristics were measured in time taken to solve diffenent
puzzles (although they are all close to 0 for 9x9 puzzles), 
maximum depth reached, and total recursive calls. 
Maximum depth signifies the deepest level of recursion before reaching
the solution. Solutions are unique and at a constant depth, which is
included here for reference. \bigskip

\begin{tabular}{l | l}
Difficulty & Solution Depth \\ \hline
Easy & 44 \\
Medium & 52 \\
Hard & 58 \\
Tough & 60 \\
Sixteen & 168 \\
\end{tabular}

\section{Different Backtracking Heuristics}
\subsection{Naive Backtracking Algorithm}
The simplest search uses no heuristics, and isn't particularly fast or
efficient. The backtracking search picks the next unassigned position
in the puzzle, and tries assigning values to that position in numerical
order. For each assignment it removes legal values from the same row, column,
and box, and then recurses with this new state. The recursive call either fails,
in which case a new value is tried, or it succeeds.

This naive algorithm does not solve the 16x16 puzzle in a reasonable
amount of time, but still solves 9x9 puzzles very quickly. \bigskip

\begin{tabular}{l || r | c | l }
Difficulty & Time (s) & Max Depth & Total Recursive Calls \\ \hline
Easy    & 0.00 & 38 & 66 \\
Medium  & 0.00 & 39 & 701 \\
Hard    & 0.01 & 51 & 8,578 \\
Tough   & 0.02 & 50 & 31,927 \\
Sixteen & n/a & n/a & n/a \\ 
\end{tabular}

\subsection{Arc3 Consistency}
This is less a modification to the backtracking algorithm, and more an extra 
step to check for failure. This algorigthm checks that every position affected
by the position in the puzzle just modified has remaining legal values. The 
more general algorithm checks that each position has a value consistent with
every value in the given position, but due to the nature of the sudoku puzzle,
this is as simple as ensuring no position has a domain size of 0 (no legal
values).

Adding the arc consistency check to the naive algorithm improves the efficiency
by a noticable amount, as it finds states with errors earlier and avoids some
unnecessary recursion. The improvement is not significant enough however to 
allow completion of the 16x16 puzzle in a reasonable amount of time. \bigskip


\begin{tabular}{l || r | c | l }
Difficulty & Time (s) & Max Depth & Total Recursive Calls \\ \hline
Easy    & 0.00 & 17 & 54 \\
Medium  & 0.00 & 37 & 257 \\
Hard    & 0.00 & 44 & 1,523 \\
Tough   & 0.01 & 44 & 12,828 \\
Sixteen & n/a & n/a & n/a \\ 
\end{tabular}

\subsection{Minimum Remaining Values}
The Minimum Remaining Values (MRV) heuristic chooses a position to test based
on the fewest legal values available to that position. By first choosing 
positions with 0 legal values, it finds grids with errors quickly and fails
early. Similarly it prioritizes positions with 1 legal value, which is naturally
either correct or incorrect, but allows for much less guesswork. 

This algorithm shows marked improvement over the naive heurisitic. For easy 
puzzles, the algorithm can pick a cell with one legal value on each recursive
step, thus never failing and backtracking (note the max failure depth reached
for the easier puzzles is 0, and the total recursive calls is equal to the
depth of the solution). This heuristic also allows for 
completion of the 16x16 puzzle.  \bigskip

\begin{tabular}{l || r | c | l }
Difficulty & Time (s) & Max Depth & Total Recursive Calls \\ \hline
Easy    & 0.00   & 0   & 44 \\
Medium  & 0.00   & 0   & 52 \\
Hard    & 0.00   & 49  & 243 \\
Tough   & 0.00   & 49  & 942 \\
Sixteen & 54.80 & 146 & 17,465,086 \\ 
\end{tabular}

\subsection{Maximum Remaining Values}
The max. remaining values heuristic is functionally identical to the
min. remaining values heurisitc, except it chooses a position with the largest
number of legal values. Used alone, this heuristic is significantly
worse than even the naive search, so much so that only the easy puzzle
completes in a reasonable amount of time. Pairing this heuristic with
other heuristics (arc consistency and most/least constraining value)
improves efficiency somewhat (reduces time to 0s), 
although it is still only able to solve the easiest puzzle. \bigskip

\begin{tabular}{l | l}
\multicolumn{2}{c}{Easy} \\
\hline
Time (s) & 4.85 \\
Max Depth & 42 \\
Total Recursive Calls & 5,700,037 \\
\end{tabular}


\subsection{Most Constraining Value}
After a position is picked, the order
of values to be tried is then chosen. Using the MCV heuristic, values are
chosen based on how many other values they rule out. This is calculated by
counting the number of positions in the same row/col with the value as a legal
value. This number is maximized for MCV, or minimized for LCV (least 
constraining value). 

When MCV is the only heuristic used, it is not particularly effective at all;
although more constraining values are picked first, they aren't any more likely
to be correct, or even fail sooner, as the results indicate. However it does 
offer some improvement when used in combination with the MRV heuristic, as is
discussed in more detail later. \bigskip

\begin{tabular}{l || r | c | l }
Difficulty & Time (s) & Max Depth & Total Recursive Calls \\ \hline
Easy    & 0.00   & 38   & 81 \\
Medium  & 0.01   & 43   & 7,820 \\
Hard    & 0.03   & 50  & 19,279 \\
Tough   & 0.06   & 50  & 34,610 \\
Sixteen & n/a & n/a  & n/a \\ 
\end{tabular}

\subsection{Least Constraining Value}
This heuristic is the opposite of the MCV heuristic; choosing a value
which constrains the fewest other positions in the puzzle.

Like MCV, when used alone, this heuristic does not offer any improvement
from the naive heuristic, and in fact actually gives worse results for the
harder puzzles than the MCV heuristic. \bigskip

\begin{tabular}{l || r | c | l }
Difficulty & Time (s) & Max Depth & Total Recursive Calls \\ \hline
Easy    & 0.00   & 19   & 68 \\
Medium  & 0.01   & 42   & 6651 \\
Hard    & 0.05   & 51  & 30,102 \\
Tough   & 0.39   & 51  & 219,377 \\
Sixteen & n/a & n/a  & n/a \\ 
\end{tabular}

\section{Combining Heuristics}
Although some heuristics are fairly powerful on their own, the most efficient
sudoku solver is created by combining different heuristics. The arc-consistency
checker offers improvement in all situations, as it simply finds errors more
quickly, and does not change the algorithm significantly. The minimum remaining
values heuristic is clearly the superior method for choosing the variable, but
the difference between the value heuristics (naive, least/most constraining) is
less pronounced, and more interesting to examine.

For conciseness, only the 16x16 puzzle is used to measure performance, as it has
the most noticable difference in time. The Maximum Remaining Values heuristic
is not being tested here due, so I will refer to Minimum Remaining Values as MRV.
ARC3 refers to the arc-consistency algorithm, LCV to the least constraining 
value heuristic, and MCV to the most constraining value heuristic. \bigskip

\begin{tabular}{l | r | r | r }
Heuristics Used  & Time (s) & Max Depth & Total Recursive Calls \\ \hline
MRV                    & 54.44 & 146 & 17,465,086 \\
MRV \&{} ARC3          & 59.19 & 145 & 15,312,511 \\
MRV \&{} LCV           & 55.21 & 146 & 17,465,086 \\
MRV \&{} MCV           & 55.47 & 154 & 17,956,284 \\
MRV \&{} ARC3 \&{} LCV & 60.93 & 150 & 15,817,945 \\
MRV \&{} ARC3 \&{} MCV & 60.98 & 153 & 15,758,792 \\
\end {tabular}

\section{Conclusion and Analysis}
We see from these results that the simple Minimum Remaining Values heuristic
is the fastest combination. However the MRV and ARC3 combination is the most
efficient by measure of total recursive calls.
It is likely that although checking for arc-consistency finds failure states
sooner and thus prune the tree more effectively, the additional computational
cost associated with the algorithm ends up not being worth it, at least in
the 16x16 puzzle. Furthermore the LCV and MCV heuristics do not add a 
significant improvement over the MRV algorithm alone. 
The MRV algorithm is fairly efficient, and quite powerful, especially with
relatively easy puzzles; if values can be narrowed down sufficiently, the
algorithm will be able to always find positions with one or zero legal values.
Thus there is little room for optimization by LCV and MCV (there are no or few
values to pick from), and the additional computation is a net negative.

Much of the performance of different algorithms is very much an artefact
of the structure of the sudoku puzzle; every position has the same number
of legal values initially, each position has exactly one correct value, and
the only way to easily constrain other positions is to pick a value for another
posisition. Thus we find the simple MRV algorithm performs the best, while the
other algorithms add more computational complexity than they return in 
efficiency.

\subsection{Additional Tests}
Here is a full list of tests, there are 18 possible combinations of these
heurisitics, all are shown either in this table or previously in this report. \smallskip


{
\centering
\begin{tabular}{m{15em} | l | r | r | l}
Heurisitics Used & Difficulty & Time (s) & Max Depth & Total Recursive Calls \\ \hline \hline
\multirow{4}{*}{Arc3 \&{} LCV}         & Easy   & 0.00 & 17 & 63 \\
                                       & Medium & 0.00 & 37 & 1,909 \\
                                       & Hard   & 0.01 & 47 & 4,343 \\
                                       & Tough  & 0.04 & 47 & 54,195 \\ \hline
\multirow{4}{*}{Arc3 \&{} MCV}         & Easy   & 0.00 & 19 & 56 \\
                                       & Medium & 0.00 & 39 & 2,508 \\
                                       & Hard   & 0.00 & 46 & 3,523 \\
                                       & Tough  & 0.01 & 44 & 13,802 \\ \hline
\multirow{5}{8em}{Arc3 \&{} Min. Remaining Values}
    & Easy   & 0.00 & 0 & 44 \\
    & Medium & 0.00 & 0 & 52 \\
    & Hard   & 0.00 & 48 & 232 \\
    & Tough  & 0.00 & 48 & 860 \\ 
    & Sixteen & 59.19 & 145 & 15,312,51 \\ \hline

\multirow{5}{8em}{Arc3 \&{} Min. Remaining Values \&{} LCV}
    & Easy   & 0.00 & 0 & 44 \\
    & Medium & 0.00 & 0 & 52 \\
    & Hard   & 0.00 & 34 & 88 \\
    & Tough  & 0.00 & 48 & 2,516 \\ 
    & Sixteen & 60.93 & 150 & 15,817,945  \\ \hline

\multirow{5}{8em}{Arc3 \&{} Min. Remaining Values \&{} MCV}
    & Easy   & 0.00 & 0 & 44 \\
    & Medium & 0.00 & 0 & 52 \\
    & Hard   & 0.00 & 48 & 224 \\
    & Tough  & 0.00 & 48 & 905 \\ 
    & Sixteen & 60.98 & 153 & 15,758,79 \\ \hline
Arc3 \&{} Max. Remaining Values 
    & Easy & 0.00 & 25 & 1,229 \\ \hline
Arc3 \&{} Max. Remaining Values \&{} LCV
    & Easy & 0.00 & 26 & 1,097 \\ \hline
Arc3 \&{} Max. Remaining Values \&{} MCV
    & Easy & 0.00 & 25 & 708 \\ \hline


\multirow{5}{8em}{Min. Remaining Values \&{} LCV}
    & Easy   & 0.00 & 0 & 44 \\
    & Medium & 0.00 & 0 & 52 \\
    & Hard   & 0.00 & 35 & 91 \\
    & Tough  & 0.00 & 49 & 2,754 \\ 
    & Sixteen & 55.21 & 146 & 17,465,08  \\ \hline

\multirow{5}{8em}{Min. Remaining Values \&{} MCV}
    & Easy   & 0.00 & 0 & 44 \\
    & Medium & 0.00 & 0 & 52 \\
    & Hard   & 0.00 & 49 & 234 \\
    & Tough  & 0.00 & 49 & 993 \\ % CHANGE THE SIXTEEN HERE TOO 
    & Sixteen & 55.47 & 154 & 17,956,284 \\ \hline
Max. Remaining Values \&{} LCV
    & Easy & 4.08 & 42 & 4,833,133 \\ \hline
Max. Remaining Values \&{} MCV
    & Easy & 1.92 & 42 & 2,258,248 \\

\end{tabular}
}
 
\end{document}
